\documentclass[a4paper, 12pt]{exam}

% Exam class is broken without this.
\makeatletter
\expandafter\providecommand\expandafter*\csname ver@framed.sty\endcsname
{2003/07/21 v0.8a Simulated by exam}
\makeatother

% Enables the use of colour.
\usepackage{xcolor}
% Syntax high-lighting for code. Requires Python's pygments.
\usepackage{minted}
% Enables the use of umlauts and other accents.
\usepackage[utf8]{inputenc}
% Diagrams.
\usepackage{tikz}
% Settings for captions, such as sideways captions.
\usepackage{caption}
% Symbols for units, like degrees and ohms.
\usepackage{gensymb}
% Latin modern fonts - better looking than the defaults.
\usepackage{lmodern}
% Allows for columns spanning multiple rows in tables.
\usepackage{multirow}
% Better looking tables, including nicer borders.
\usepackage{booktabs}
% More math symbols.
\usepackage{amssymb}
% More math layouts, equation arrays, etc.
\usepackage{amsmath}
% More math fonts, like mathbb.
\usepackage{amsfonts}
% More theorem environments.
\usepackage{amsthm}
% More column formats for tables.
\usepackage{array}
% Adjust the sizes of box environments.
\usepackage{adjustbox}
% Better looking single quotes in verbatim and minted environments.
\usepackage{upquote}
% URLs.
\usepackage[hidelinks]{hyperref}
% Better blank space decisions.
\usepackage{xspace}
% Better looking tikz trees.
\usepackage{forest}
% For plotting.
\usepackage{pgfplots}

% Various tikz libraries.
\usetikzlibrary{mindmap}
\usetikzlibrary{shadows}
\usetikzlibrary{arrows}
\usetikzlibrary{positioning}
\usetikzlibrary{chains}
\usetikzlibrary{fit}
\usetikzlibrary{shapes}
\usetikzlibrary{decorations.markings}
\usetikzlibrary{calc}
\usetikzlibrary{arrows.meta}

% GMIT colours.
\definecolor{gmitblue}{RGB}{20,134,225}
\definecolor{gmitred}{RGB}{220,20,60}
\definecolor{gmitgrey}{RGB}{67,67,67}

% Rename Bibliography to a smaller "Refereces".
\renewcommand{\refname}{\selectfont\normalsize References} 

% Stop minted high-lighting errors.
\makeatletter
\expandafter\def\csname PYGdefault@tok@err\endcsname{\def\PYGdefault@bc##1{{\strut ##1}}}
\makeatother

% Set the header and footer.
\pagestyle{headandfoot}
\header{\textbf{Problem Sheet: Difficult problems}}{}{Theory of Algorithms}
\footer{}{Page \thepage\ of \numpages}{}

% Change some things to do with marks.
\marksnotpoints
\pointsinrightmargin

% Empty cover page.
\begin{coverpages}
\end{coverpages}

\begin{document}

\printanswers

\noindent
This problem sheet is about difficult computational problems~\cite{sipserbook}.


\begin{questions}

\question
  Explain what is meant by SUBSETSUM, and the decision problem related to it.
  \begin{solution}
    SUBSETSUM is the set of all sets of integers that have a subset the sum of all whose elements is zero.
    $$ \textrm{SUBSETSUM} = \{ S \mid S \subseteq \mathbb{Z}, \exists T \subseteq S: \sum_{i \in T} i = 0 \} $$
  \end{solution}


\question
  Write a function in Racket that decides SUBSETSUM.
  \begin{solution}
    \begin{minted}{scheme}
(define (subsetsum l)
  (not (null? (filter
    (lambda (y) (= 0 (apply + y)))
    (filter
      (lambda (x) (not (null? x)))
      (combinations l))))))
    \end{minted}
  \end{solution}


\question
  Write a function in Java that decides SUBSETSUM.
  \begin{solution}
    \begin{minted}{java}
// From: www.geeksforgeeks.org/dynamic-programming-subset-sum-problem
static boolean isSubsetSum(int set[], int n, int sum) {
  if (sum == 0)
    return true;
  if (n == 0 && sum != 0)
    return false;

  if (set[n-1] > sum)
    return isSubsetSum(set, n-1, sum);
  
  return isSubsetSum(set, n-1, sum) || 
            isSubsetSum(set, n-1, sum-set[n-1]);
}
    \end{minted}
  \end{solution}


\question
  Explain the integer factorisation problem and why it is important in modern cryptography.
  \begin{solution}
    The integer factorisation problem is the problem of factorising a natural number (greater than one) into a product of prime numbers.
    Every such number is a distinct product of primes, up to re-organsing the order of multiplication.
    For instance, the only way to factorise the number 561 into a product of primes is $3 \times 11 \times 17$.

    The complexity of integer factorisation is not know.
    No polynomial time algorithm has been developed, and it is not known if the problem is NP-complete.
    Modern asymmetric key cryptgraphic systems rely on the factorisation of a product of two similar-sized prime numbers not being efficiently possible.
    The fact that PRIMES is in P means that we can find two similar sized prime numbers efficiently.
    Modern cryptography exploits these two facts.
  \end{solution}


\question
  Explain what is meant by SAT.
  \begin{solution}
    SAT is the set of all Boolean formulas where some setting of the constituent Boolean variables satisfies the formula.
    For instance, the formula $a \land (b \lor c)$ is satisfiable (set $a$ and $b$ to true), but $a \land \neg a$ is not.
  \end{solution}

\question
  Explain the 3-SAT decision problem.
  \begin{solution}
   3-SAT is the set of all satisfiable Boolean formulas in Conjunctive Normal Form where each clause contains three literals.
   The decision problem related to 3-SAT is the problem of deciding 3-SAT.
   Sometimes this is called the 3-SAT decision problem.
  \end{solution}

\question
  Explain the terms conjunctive normal form and disjunctive normal form.
  \begin{solution}
    A Boolean formula is in Conjunctive Normal Form (CNF) if it is a conjunction of disjunctions.
    That is, it is a series of clauses with ands between them, where each clauses is a series of literals with ors between them.
    A literal is a variable or its negation.
    For example, the following formula is in CNF.
      $$ (a \lor b \lor neg c) \land (\neg a \lor c) $$
    
    Disjunctive Normal Form (DNF) is a disjunction of conjunctions -- the ors are outside the brackets, and the ands are inside.
    For example, the following is a formula in DNF.
      $$ (a \land b) \lor (\neg a \land \neg b) $$
  \end{solution}


\question
  Convert the following expressions to Conjunctive Normal Form.
  \begin{parts}
    \part $a \vee b$
    \part $a \wedge b$
    \part $((a \wedge b) \vee ( \neg b \wedge c)) \vee  \neg d$
    \part $(a  \wedge   b) \vee (c  \wedge  d)$
    \part $(a \vee b)  \wedge  (c \vee d)$
  \end{parts}
  \begin{solution}
  \begin{parts}
    \part $a \vee b$
    \part $a \wedge b$
    \part $(a \vee \neg b \vee \neg d) \wedge (b \vee c \vee \neg d)$
    \part $(a\vee c) \wedge (a\vee d) \wedge (b\vee c) \wedge (b\vee d)$
    \part $(a \vee b)  \wedge  (c \vee d)$
  \end{parts}
  \end{solution}

\question
  Convert the following expressions to Disjunctive Normal Form.
  \begin{parts}
    \part $a \vee b$
    \part $a \wedge b$
    \part $((a \wedge b) \vee ( \neg b \wedge c)) \vee  \neg d$
    \part $(a \wedge b) \vee (c \wedge d)$
    \part $(a \vee b)  \wedge  (c \vee d)$
  \end{parts}
  \begin{solution}
  \begin{parts}
    \part $a \vee b$
    \part $a \wedge b$
    \part $(a \wedge b) \vee (\neg b \wedge c) \vee \neg d$
    \part $(a  \wedge  b) \vee (c  \wedge  d)$
    \part $(a \wedge c)\vee(a \wedge d)\vee(b \wedge c)\vee(b \wedge d)$
  \end{parts}
  \end{solution}


\question
  Determine if there is a setting of the variables in the following expression that makes the evaluation of the expression true.
  \begin{parts}
    \part $a \vee b$
    \part $a \wedge b$
    \part $((a \wedge b) \vee ( \neg b \wedge c)) \vee  \neg d$
    \part $(a  \wedge   b) \vee (c  \wedge  d)$
    \part $(a \vee b)  \wedge  (c \vee d)$
  \end{parts}
  \begin{solution}
  \begin{parts}
    \part $(a,b) = (1,1)$
    \part $(a,b) = (1,1)$
    \part $(a,b,c,d) = (1,1,1,0)$
    \part $(a,b,c,d) = (1,1,1,0)$
    \part $(a,b,c,d) = (1,1,1,0)$
  \end{parts}
  \end{solution}


\question
  Explain how to prove that a decision problem is NP-complete.
  \begin{solution}
    A decision problem is NP-complete if it is noth NP-hard and in NP itself.
    Thus, to show a decision problem is NP-complete we must show that it satisfies both of these properties.
    A decision problem is in NP if an instance can be verified in polynomial time, which is the same as being decidable in polynomial time by a non-determinisitic Turing machine.

    A problem is NP-hard if every problem in NP can be reduced to it in polynomial time.
    A reduction from problem B to problem A is an algorithm for converting any instance of B to an instance of A.
    The easiest way to show a problem in NP is NP-complete is to reduce an NP-complete problem to it in polynomial time to .
  \end{solution}


\question
  Prove that 3-SAT is NP-complete. You may assume that SAT is NP-complete.
  \begin{solution}
    The set 3-SAT is a subset of SAT, and SAT is in NP.
    Thus, 3-SAT is in NP.

    We can convert an instance of SAT to an instance of 3-SAT in the following way.
    First convert the formula to CNF.
    Then convert any one-literal clause $a$ of the SAT formula to the following 3-SAT formula:
      $$(a \lor u_1 \lor u_2) \land (a \lor u_1 \lor \neg u_2) \land (a \lor \neg u_1 \lor u_2) \land (a \lor \neg u_1 \lor \neg u_2).$$
    Convert any two-variable clause $a \lor b$ to:
      $$(a \lor b \lor u_1) \land (a \lor b \lor \neg u_1)).$$
    Any three-variable clause can be left as-is.
    Convert any clause with more than three variables $a \lor b \lor c \lor \ldots$ to:
      $$(a \lor b \lor u_1) \land (c \lor \neg u_1 \lor u_2)) \land \ldots \land (i \lor \neg u_{n-4} \lor u_{n-3}) \land (j \lor k \lor \neg u_{n-3}).$$
    Now the formula is in 3-CNF.
    This reduction is polynomial in time, because the output is at most three times as long as the input.
  \end{solution}



\end{questions}
\bibliographystyle{plain}
\bibliography{bibliography}
\end{document}
