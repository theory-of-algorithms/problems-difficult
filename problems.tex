\documentclass[addpoints,12pt]{exam}

\makeatletter
\expandafter\providecommand\expandafter*\csname ver@framed.sty\endcsname
{2003/07/21 v0.8a Simulated by exam}
\makeatother

\usepackage{xcolor}
\usepackage{minted}
\usepackage[utf8]{inputenc}
\usepackage{tikz}
\usepackage{caption}
\usepackage{gensymb}
\usepackage{lmodern}
\usepackage{multirow}
\usepackage{booktabs}
\usepackage{amssymb}
\usepackage{array}
\usepackage{adjustbox}
\usepackage{upquote}
\usepackage{amsmath}
\usepackage[hidelinks]{hyperref}
\usetikzlibrary{mindmap,shadows, shapes, arrows, positioning}

\tikzstyle{rect} = [rectangle, fill=ProcessBlue, text width=4.5em, text centered, minimum height=4em, rounded corners]
\tikzstyle{line} = [draw, ->, very thick]
\tikzstyle{oval} = [ellipse, fill=SeaGreen, text width=5em, text centered]

\newcolumntype{x}[1]{>{\centering\arraybackslash\hspace{0pt}}p{#1}}

\renewcommand{\refname}{\selectfont\normalsize References} 
\pagestyle{headandfoot}

\header{\textbf{Problem Sheet: Difficult problems}}{}{Theory of Algorithms}
\footer{}{Page \thepage\ of \numpages}{}
\marksnotpoints
\pointsinrightmargin

\begin{coverpages}
\end{coverpages}

\begin{document}

%\printanswers

\noindent
This problem sheet is about difficult computational problems~\cite{sipserbook}.


\begin{questions}


\question
  Write a Racket function that decides PRIMES.
  \begin{solution}
    \begin{minted}{java}

    \end{minted}
  \end{solution}

\question
  Write a Java function that decides PRIMES.
  \begin{solution}
    \begin{minted}{java}

    \end{minted}
  \end{solution}


\question
  Explain what is meant by SUBSETSUM, and the decision problem related to it.

\question
  Write a function in Racket that decides SUBSETSUM.

\question
  Write a function in Java that decides SUBSETSUM.

\question
  Explain the integer factorisation problem and why it is important in modern cryptography.

\question
  Explain what is meant by SAT.

  \begin{solution}
  As in notes.
  \end{solution}

\question
  Explain the 3-SAT decision problem.
  \begin{solution}
  As in notes.
  \end{solution}


\question
Explain the subset sum problem, and the brute-force approach to solving it.



\question
Explain the INTFAC problem.

\question
Explain the SAT problem.
\begin{solution}
As in notes.
\end{solution}

\question
Explain the 3-SAT problem.
\begin{solution}
As in notes.
\end{solution}



\question
Explain the terms conjunctive normal form and disjunctive normal form.
\begin{solution}
As in notes.
\end{solution}


\question
Convert the following expressions to Conjunctive Normal Form.
\begin{parts}
  \part $a \vee b$
  \part $a \wedge b$
  \part $((a \wedge b) \vee ( \neg b \wedge c)) \vee  \neg d$
  \part $(a  \wedge   b) \vee (c  \wedge  d)$
  \part $(a \vee b)  \wedge  (c \vee d)$
\end{parts}
\begin{solution}
\begin{parts}
  \part $a \vee b$
  \part $a \wedge b$
  \part $(a \vee \neg b \vee \neg d) \wedge (b \vee c \vee \neg d)$
  \part $(a\vee c) \wedge (a\vee d) \wedge (b\vee c) \wedge (b\vee d)$
  \part $(a \vee b)  \wedge  (c \vee d)$
\end{parts}
\end{solution}

\question
Convert the following expressions to Disjunctive Normal Form.
\begin{parts}
  \part $a \vee b$
  \part $a \wedge b$
  \part $((a \wedge b) \vee ( \neg b \wedge c)) \vee  \neg d$
  \part $(a \wedge b) \vee (c \wedge d)$
  \part $(a \vee b)  \wedge  (c \vee d)$
\end{parts}
\begin{solution}
\begin{parts}
  \part $a \vee b$
  \part $a \wedge b$
  \part $(a \wedge b) \vee (\neg b \wedge c) \vee \neg d$
  \part $(a  \wedge  b) \vee (c  \wedge  d)$
  \part $(a \wedge c)\vee(a \wedge d)\vee(b \wedge c)\vee(b \wedge d)$
\end{parts}
\end{solution}


\question
Determine if there is a setting of the variables in the following expression that makes the evaluation of the expression true.
\begin{parts}
  \part $a \vee b$
  \part $a \wedge b$
  \part $((a \wedge b) \vee ( \neg b \wedge c)) \vee  \neg d$
  \part $(a  \wedge   b) \vee (c  \wedge  d)$
  \part $(a \vee b)  \wedge  (c \vee d)$
\end{parts}

\begin{solution}
\begin{parts}
  \part $(a,b) = (1,1)$
  \part $(a,b) = (1,1)$
  \part $(a,b,c,d) = (1,1,1,0)$
  \part $(a,b,c,d) = (1,1,1,0)$
  \part $(a,b,c,d) = (1,1,1,0)$
\end{parts}
\end{solution}


\question
Explain how to prove that a problem is NP-complete.
\begin{solution}
As in notes.
\end{solution}


\question
Prove that 3-SAT is NP-complete. You may assume that SAT is NP-complete.
\begin{solution}
As in notes.
\end{solution}



\end{questions}



\bibliographystyle{plain}
\bibliography{bibliography}
\end{document}
