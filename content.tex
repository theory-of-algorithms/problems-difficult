%!TEX root = problems.tex

\printanswers

%\noindent
%This is the introduction~\cite{knuthwebsite}.

\begin{questions}


\question
Explain the terms conjunctive normal form and disjunctive normal form.
\begin{solution}
As in notes.
\end{solution}


\question
Convert the following expressions to Conjunctive Normal Form.
\begin{parts}
  \part $a \vee b$
  \part $a \wedge b$
  \part $((a \wedge b) \vee ( \neg b \wedge c)) \vee  \neg d$
  \part $(a  \wedge   b) \vee (c  \wedge  d)$
  \part $(a \vee b)  \wedge  (c \vee d)$
\end{parts}
\begin{solution}
\begin{parts}
  \part $a \vee b$
  \part $a \wedge b$
  \part $(a \vee \neg b \vee \neg d) \wedge (b \vee c \vee \neg d)$
  \part $(a\vee c) \wedge (a\vee d) \wedge (b\vee c) \wedge (b\vee d)$
  \part $(a \vee b)  \wedge  (c \vee d)$
\end{parts}
\end{solution}

\question
Convert the following expressions to Disjunctive Normal Form.
\begin{parts}
  \part $a \vee b$
  \part $a \wedge b$
  \part $((a \wedge b) \vee ( \neg b \wedge c)) \vee  \neg d$
  \part $(a \wedge b) \vee (c \wedge d)$
  \part $(a \vee b)  \wedge  (c \vee d)$
\end{parts}
\begin{solution}
\begin{parts}
  \part $a \vee b$
  \part $a \wedge b$
  \part $(a \wedge b) \vee (\neg b \wedge c) \vee \neg d$
  \part $(a  \wedge  b) \vee (c  \wedge  d)$
  \part $(a \wedge c)\vee(a \wedge d)\vee(b \wedge c)\vee(b \wedge d)$
\end{parts}
\end{solution}


\question
Determine if there is a setting of the variables in the following expression that makes the evaluation of the expression true.
\begin{parts}
  \part $a \vee b$
  \part $a \wedge b$
  \part $((a \wedge b) \vee ( \neg b \wedge c)) \vee  \neg d$
  \part $(a  \wedge   b) \vee (c  \wedge  d)$
  \part $(a \vee b)  \wedge  (c \vee d)$
\end{parts}

\begin{solution}
\begin{parts}
  \part $(a,b) = (1,1)$
  \part $(a,b) = (1,1)$
  \part $(a,b,c,d) = (1,1,1,0)$
  \part $(a,b,c,d) = (1,1,1,0)$
  \part $(a,b,c,d) = (1,1,1,0)$
\end{parts}
\end{solution}


\question
Explain the SAT problem.
\begin{solution}
As in notes.
\end{solution}

\question
Explain the 3-SAT problem.
\begin{solution}
As in notes.
\end{solution}

\question
Explain how to prove that a problem is NP-complete.
\begin{solution}
As in notes.
\end{solution}


\question
Prove that 3-SAT is NP-complete. You may assume that SAT is NP-complete.
\begin{solution}
As in notes.
\end{solution}

\end{questions}
